% =============================================================================
%  saterus's slides for a presentation on Haskell
% =============================================================================

\documentclass{beamer}                  % beamer class, good for slides
\usepackage[T1]{fontenc}                % use a modern scalable font
\usepackage{lmodern}                    % Latin Modern font
\usepackage[T1]{tipa}                   % pronunciation symbols
\usepackage{graphicx}                   % include pictures
\usepackage{xifthen}                    % \ifthenelse
\usepackage{framed}                     % framed environment
\usepackage{cancel}                     % \cancel in math
\usepackage{varioref}                   % \vpageref cross referencing
\usepackage{listings}                   % code formatting

% - - - - - - - - - - - - - - - - - - - - - - - - - - - - - - - - - - - - - - -
%  Theming
% - - - - - - - - - - - - - - - - - - - - - - - - - - - - - - - - - - - - - - -
%
% going for a simple, pastel look here
%

\usetheme{default}                       % default beamer theme (clean, empty)
\beamertemplatenavigationsymbolsempty    % remove nav symbols

\definecolor{fore}{RGB}{249,242,215}     % foreground color, off-white
\definecolor{back}{RGB}{51,51,51}        % background color, gray
\definecolor{title}{RGB}{96,148,219}     % title color, pastel blue
\definecolor{keywords}{RGB}{255,0,90}    % code keyword color, pastel pink
\definecolor{comments}{RGB}{0,179,113}   % code comment color, pastel green
\definecolor{item}{RGB}{96,148,219}      % title color, pastel blue

\setbeamercolor{titlelike}{fg=title}          % titles use title color
\setbeamercolor{normal text}{fg=fore,bg=back} % text is fore on back
\setbeamercolor{block title}{fg=comments}     % block titles are comment color
\setbeamercolor{section in toc}{fg=comments}  % toc uses title color
\setbeamercolor{item}{fg=item}                % itemize symbol color


% - - - - - - - - - - - - - - - - - - - - - - - - - - - - - - - - - - - - - - -
%  code listing
% - - - - - - - - - - - - - - - - - - - - - - - - - - - - - - - - - - - - - - -
%
% beamer and listings don't play super well together
% best to define code *outside* of a frame and then call it, like so:
%
% \defverbatim[colored]\code{           % define code out of frame
% \begin{lstlisting}
%  -- code goes here --
% \end{lstlisting}
% } % end of stored code
%
% \begin{frame}
% \frametitle{title}
% \code % call code in frame
% \end{frame}
%

\lstset{                                % listings settings
	language=[LaTeX]TeX,                % LaTeX by default
	upquote=false,                      % do NOT use "
	tabsize=4,                          % tabs are 4 spaces
	basicstyle=\ttfamily,               % code is typewriter-style
	keywordstyle=\color{keywords},      % keywords use keyword color
	commentstyle=\color{comments}\emph, % comments use comment color
}

% - - - - - - - - - - - - - - - - - - - - - - - - - - - - - - - - - - - - - - -
%  section slide command
% - - - - - - - - - - - - - - - - - - - - - - - - - - - - - - - - - - - - - - -
%
% new command to draw a simple line.  it will be as long as available,
% and 1pt wide
%

\newcommand{\srule}{
	\rule{\textwidth}{1pt}\\
}

% - - - - - - - - - - - - - - - - - - - - - - - - - - - - - - - - - - - - - - -
%  custom slide environment
% - - - - - - - - - - - - - - - - - - - - - - - - - - - - - - - - - - - - - - -
%
% essentially frame, but automatically adds section and possibly subsection as
% title.  also increases default font size
%
% method to detect subsection presence is kind of hacky:
% find it's width. if it has any width, it exists.
%

% variable to hold subsection's width
\newlength{\subsecwidth}

% slide environment - frame, plus automatic title
\newenvironment{slide}{
	\begin{frame} % frame
	\settowidth{\subsecwidth}{\insertsubsection} % find subsection width
	\ifthenelse{\dimtest{\subsecwidth}{<}{1pt}}{ % no subsection
		\frametitle{\insertsection\\             % insert *just* section
		\vspace{-1ex}                            % move next line up a bit
		\color{fore}\srule                       % pretty line
		\par                                     % remove excess spacing
		\vspace{-3ex}                            % remove excess spacing
		}
	}{                                           % subsection exists
		\frametitle{\insertsection\ -- \insertsubsection\\ % sec - subsec
		\vspace{-1ex}                            % move next line up a bit
		\color{fore}\srule                       % pretty line
		\par                                     % remove excess spacing
		\vspace{-3ex}                            % remove excess spacing
		}
	}
	\Large                                       % make font in slide Large
}{
	\end{frame}
}

% - - - - - - - - - - - - - - - - - - - - - - - - - - - - - - - - - - - - - - -
%  code result environment
% - - - - - - - - - - - - - - - - - - - - - - - - - - - - - - - - - - - - - - -
%
% code is often paired with there result, but has to be entered outside of
% frame.  use code from outside frame (\code), paired with result
%

\newenvironment{coderesult}{
	\begin{block}{Code}      % block, called Code
		\code                % print \code
	\end{block}
	\begin{block}{Result}    % block, called Result
}{                           % result argument is here
	\end{block}
}

% - - - - - - - - - - - - - - - - - - - - - - - - - - - - - - - - - - - - - - -
%  section slide command
% - - - - - - - - - - - - - - - - - - - - - - - - - - - - - - - - - - - - - - -
%
% simple command to both set the section and display a lone frame indicating
% the new section
%

\newcommand{\titleslide}[1]{
	\section{#1}             % set the section based on argument
	\begin{slide}
		\begin{center}
			\color{comments}
			\Huge            % Huge font size
			#1               % print new section's title
		\end{center}
	\end{slide}
}

% - - - - - - - - - - - - - - - - - - - - - - - - - - - - - - - - - - - - - - -
%  formatting commands
% - - - - - - - - - - - - - - - - - - - - - - - - - - - - - - - - - - - - - - -

\newcommand{\mediatitle}[1]{\textit{#1}}  % media titles should be italicized
\newcommand{\forlang}[1]{\textit{#1}}     % foreign languages should be ital

% - - - - - - - - - - - - - - - - - - - - - - - - - - - - - - - - - - - - - - -
%  formatting commands
% - - - - - - - - - - - - - - - - - - - - - - - - - - - - - - - - - - - - - - -

\renewcommand{\thefootnote}{\fnsymbol{footnote}} % fancy symbols for footnotes

% - - - - - - - - - - - - - - - - - - - - - - - - - - - - - - - - - - - - - - -
%  title block
% - - - - - - - - - - - - - - - - - - - - - - - - - - - - - - - - - - - - - - -

\title{Introduction to Weird Functional Languages with Haskell Edition}    % title
\author{
	Alex ``saterus'' Burkhart\\          % author
	The Ohio State University\\         % university
	Open Source Club                    % club
}
\date{2011-05-19}                       % date

% =============================================================================
%  actual document begins here
% =============================================================================

\begin{document}                        % settings end, content begins

\begin{frame}                           % title slide
	\srule                              % pretty line
	\titlepage                          % title page (title, author, date)
	\srule                              % pretty line
\end{frame}

\begin{frame}                           % Table of contents slide
	\begin{center}
		\srule                          % pretty line
		\vspace{1ex}
		\color{title} \inserttitle\\\color{fore} Table of Contents
		\srule                          % pretty line
	\end{center}
	\begin{columns}                     % break into two columns
		\begin{column}{.5\textwidth}    % first column, 1/2 page width
			\tableofcontents[sections={1-3}] % first three sections of ToC
		\end{column}
		\begin{column}{.5\textwidth}    % second column, 1/2 page width
			\tableofcontents[sections={4-6}] % last three sections of ToC
		\end{column}
	\end{columns}
\end{frame}

% - - - - - - - - - - - - - - - - - - - - - - - - - - - - - - - - - - - - - - -
%  Intro to FP
% - - - - - - - - - - - - - - - - - - - - - - - - - - - - - - - - - - - - - - -

\titleslide{Functional Programming}

\subsection{The Functional Paradigm}
\begin{slide}
  \begin{itemize}
    \item Declarative
    \item First Class Functions
    \item Pure Functions
    \item Immutibility
    \item Parallelism
  \end{itemize}
\end{slide}

\begin{slide}
  Declarative
  \begin{itemize}
    \item ``What'' instead of ``How''
    \item Smart Compilers
    \item Safety and Correctness
  \end{itemize}
\end{slide}

\begin{slide}
  First Class Functions
  \begin{itemize}
    \item Functions as Data Structures
    \item Use Functions as Arguments to Other Functions
    \item Abstract Common Patterns
  \end{itemize}
\end{slide}

\begin{slide}
  Pure Functions
  \begin{itemize}
    \item No I/O or Modification of State
    \item Consistent, Predictible Results
    \item Safety
    \item Optimization
  \end{itemize}
\end{slide}

\begin{slide}
  Immutability
  \begin{itemize}
    \item Can't Change Data
    \item Immutable Structures can be Shared
    \item Reusing List Links
  \end{itemize}
\end{slide}

\begin{slide}
  Parallelism
  \begin{itemize}
    \item Locks Unnecessary
    \item Order of Execution negotiated by Compiler
  \end{itemize}
\end{slide}

\begin{slide}
  Extra Stuff
  \begin{itemize}
    \item Code Generation with Lisp
    \item Type Safety with Haskell and OCaml
    \item Massive Paralellism with Erlang
    \item Haskell, Erlang, OCaml are all \textit{fast}
  \end{itemize}
\end{slide}


\titleslide{Haskell}

\subsection{Basic Syntax}

\begin{slide}
  Basic Data Structures
  \begin{itemize}
    \item Bool, Char, Numbers, List, String
    \item True and False
    \item 'a' through 'z' and more
    \item Ratios and Arbitrary Sized
    \item \code [1, 2, 3]
  \end{itemize}
\end{slide}

\begin{slide}
  Function Signatures
  \begin{itemize}
    \item 12 :: Int
    \item \code [1, 2, 3] :: [Integer]
    \item sum :: [Integer] -> Integer
    \item \code ['f', 'o', 'o'] :: String
  \end{itemize}
\end{slide}

\begin{slide}
  Algebraic Data Types
  \begin{itemize}
    \item data Bool = True | False
    \item data Int = -2147483648 | -2147483647 | ... | -1 | 0 | 1 | 2 | ... | 2147483647
    \item data Color = Red | Green | Blue
    \item data TrafficLight = Light String String Color
    \item data Maybe a = Nothing | Just a
  \end{itemize}
\end{slide}

\begin{slide}
  Function Definitions
  \begin{itemize}
    \item
      \code
      even :: Integer -> Integer -> Bool\\
      even x = (mod x 2) == 0
    \item
      odd :: Integer -> Integer -> Bool\\
      odd x = not (even x)
    \item
      doubleMe x = x + x
  \end{itemize}
\end{slide}

\subsection{Typeclasses}
\begin{slide}
  The Problem
  \begin{itemize}
    \item Function Scope
    \item Equality for the Color type\\
      \code
      (==) :: Color -> Color -> Bool\\
      (==) colorA colorB = ...
  \end{itemize}
\end{slide}

\begin{slide}
  The Solution
  \begin{itemize}
    \item Ad-hoc polymorphic interfaces
    \item
      \code
      class Eq a where\\
      (==) :: a -> a -> Bool\\
      (/=) :: a -> a -> Bool
    \item
      \code
      instance Eq Color where\\
        (==) :: Color -> Color -> Bool\\
        (==) a b = ...

        (/=) :: Color -> Color -> Bool\\
        (/=) a b = not (a == b)
  \end{itemize}
\end{slide}

\begin{slide}
  The Win
  \begin{itemize}
    \item Typeclass Deriving
    \item
      \code
      data Color = Red | Green | Blue\\
      deriving (Eq, Show, Read)
    \item Automatically derive instances of Read, Show, Bounded, Enum, Eq, and Ord.
  \end{itemize}
\end{slide}

\begin{slide}
  Function Signatures Updated
  \begin{itemize}
    \item 12 :: (Num a) => a
    \item \code [1,2,3] :: (Num a) => [a]
    \item even :: Integral a => a -> a -> Bool
    \item sort :: (Ord a) => [a] -> [a]
  \end{itemize}
\end{slide}

\subsection{Pattern Matching}
\begin{slide}
  Avoid lot's of manual == comparisons
  \begin{itemize}
    \item data Color = Red | Green | Blue
    \item Implement Show
    \item Implement (==)
  \end{itemize}
\end{slide}

\begin{slide}
  Avoid lot's of manual == comparisons
  \begin{itemize}
    \item data Color = Red | Green | Blue
      \item
        show :: Color -> String\\
        show Red = ``Red''\\
        show Green = ``Green''\\
        show Blue = ``Blue''
  \end{itemize}
\end{slide}

\begin{slide}
  Avoid lot's of manual == comparisons
  \begin{itemize}
    \item data Color = Red | Green | Blue
    \item
        (==) :: Color -> Color -> Bool\\
        (==) Red Red     = True\\
        (==) Green Green = True\\
        (==) Blue Blue   = True\\
        (==) \_ \_         = False
  \end{itemize}
\end{slide}

\begin{slide}
  data Maybe a = Nothing | Just a
  \begin{itemize}
    \item val1 = Just 73
    \item val2 = Nothing
    \item possiblyDouble :: Maybe Int -> Maybe Int\\
      possiblyDouble x = undefined
  \end{itemize}
\end{slide}

\begin{slide}
  data Maybe a = Nothing | Just a
  \begin{itemize}
    \item val1 = Just 73
    \item val2 = Nothing
    \item possiblyDouble :: Maybe Int -> Maybe Int\\
      possiblyDouble Nothing = Nothing\\
      possiblyDouble (Just x) = x + x
  \end{itemize}
\end{slide}

\begin{slide}
  Matching inside Lists
  \begin{itemize}
    \item myList = [1,2,3,4,5]
    \item sum :: [Int] -> Int -> Int\\
          sum [] acc     = acc\\
          sum [x:xs] acc = sum xs (acc + x)
  \end{itemize}
\end{slide}

\begin{slide}
  Matching with Case Statements
  \begin{itemize}
    \item val1 = Just 73
    \item val2 = Nothing
    \item possiblyDouble :: Maybe Int -> Maybe Int\\
      possibleDouble x = case x of\\
      Nothing -> Nothing\\
      (Just x) -> x + x
  \end{itemize}
\end{slide}

\begin{slide}
  Guards
  \begin{itemize}
    \item val1 = 5
    \item val2 = 9001
    \item doubleSmall :: Int -> Int\\
      doubleSmall x\\
      | x <= 9000 = x + x\\
      | otherwise = x
  \end{itemize}
\end{slide}

\subsection{Think Functionally}
\begin{slide}
  Recusion with Lists
  \begin{itemize}
    \item myList = [1,2,3,4,5]
    \item sum :: [Int] -> Int -> Int\\
          sum [] acc     = acc\\
          sum [x:xs] acc = sum xs (acc + x)
  \end{itemize}
\end{slide}

\begin{slide}
  Functions as Arguments
  \begin{itemize}
    \item myList = [1,2,3,4,5]
    \item double :: Int -> Int
    \item map :: (a -> b) -> [a] -> [b]
  \end{itemize}
\end{slide}

\begin{slide}
  Functions as Arguments
  \begin{itemize}
    \item example = ``things''
    \item toUpper :: Char -> Char
    \item map :: (a -> b) -> [a] -> [b]
  \end{itemize}
\end{slide}

\begin{slide}
  Functions as Arguments
  \begin{itemize}
    \item myList = [1,2,3,4,5]
    \item foldl :: (a -> b -> a) -> a -> [b] -> a
    \item sum = ?
  \end{itemize}
\end{slide}

\begin{slide}
  Higher Order Functions
  \begin{itemize}
    \item map :: (a -> b) -> [a] -> [b]
    \item filter :: (a -> Bool) -> [a] -> [a]
    \item zipWith :: (a -> b -> c) -> [a] -> [b] -> [c]
    \item foldl :: (a -> b -> a) -> a -> [b] -> a
    \item foldl1 :: (a -> a -> a) -> [a] -> a
    \item foldr :: (a -> b -> b) -> b -> [a] -> b
    \item foldr1 :: (a -> a -> a) -> [a] -> a
    \item any :: (a -> Bool) -> [a] -> Bool
    \item all :: (a -> Bool) -> [a] -> Bool
  \end{itemize}
\end{slide}

\begin{slide}
  Lambdas
  \begin{itemize}
    \item myList = [1,2,3,4,5]
    \item triple = map (\x -> x + x + x) myList
    \item divisibleThree = filter (\x -> (mod x 3) == 0) myList
  \end{itemize}
\end{slide}

\begin{slide}
  Currying
  \begin{itemize}
    \item
      ghci> :t (+)\\
      (+) :: Num a => a -> a -> a\\
      ghci> :t (+ 1)\\
      (+ 1) :: Num a => a -> a\\
    \item
      applyTwice :: (a -> a) -> a -> a\\
      applyTwice f x = f (f x)
  \end{itemize}
\end{slide}

\begin{slide}
  Flip
  \begin{itemize}
    \item flip :: (a -> b -> c) -> b -> a -> c
    \item
  \end{itemize}
\end{slide}



% - - - - - - - - - - - - - - - - - - - - - - - - - - - - - - - - - - - - - - -
% Where to go to learn more
% - - - - - - - - - - - - - - - - - - - - - - - - - - - - - - - - - - - - - - -

\renewcommand{\thefootnote}{\arabic{footnote}} % numbered footnotes
\setcounter{footnote}{0}                       % reset footnote numbering
\titleslide{Where to go to Learn More}
\begin{slide}
	\begin{itemize}
		\normalsize
		\item Wikibooks
		\footnote{
			HTML:\\ \url{https://secure.wikimedia.org/wikibooks/en/wiki/LaTeX}
		}
		\footnote{
			PDF:\\ \url{http://upload.wikimedia.org/wikipedia/commons/2/2d/LaTeX.pdf}
		}
		\item The Not So Short Introduction to \LaTeXe
		\footnote{
			PDF:\\ \url{http://www.ctan.org/tex-archive/info/lshort/english/lshort.pdf}
		}
		\item \mediatitle{The \TeX book} by Donald E. Knuth (Plain \TeX)
		\footnote{
		 	\TeX:\\
			\url{http://tex.loria.fr/general/texbook.tex}
		}
		\footnote{
		 \$32.20:
			\url{http://www.amazon.com/TeXbook-Donald-Knuth/dp/0201134489}
		}
		\item \mediatitle{\LaTeX: A Document Preparation System} by Leslie Lamport
		\footnote{
		 \$29.99:
			\url{http://www.amazon.com/LaTeX-Document-Preparation-System-2nd/dp/0201529831}
		}
		\item \mediatitle{The \LaTeX\ Companion}
		\footnote{
		 \$54.56:
			\url{http://www.amazon.com/gp/product/0201362996}
		}
	\end{itemize}
\end{slide}

\end{document}
\section{\LaTeX\ gotchas}
\section{Where to go from here}

latex   - 219
tex     - 5
context - 9

Beamer user guide - 240p
pgf/tikz manual -


\definecolor{item}{RGB}{253,198,137}     % item color, pastel peach
