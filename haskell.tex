% =============================================================================
%  saterus's slides for a presentation on Haskell
% =============================================================================

\documentclass{beamer}                  % beamer class, good for slides
\usepackage[T1]{fontenc}                % use a modern scalable font
\usepackage{lmodern}                    % Latin Modern font
\usepackage[T1]{tipa}                   % pronunciation symbols
\usepackage{graphicx}                   % include pictures
\usepackage{xifthen}                    % \ifthenelse
\usepackage{framed}                     % framed environment
\usepackage{cancel}                     % \cancel in math
\usepackage{varioref}                   % \vpageref cross referencing
\usepackage{listings}                   % code formatting

% - - - - - - - - - - - - - - - - - - - - - - - - - - - - - - - - - - - - - - -
%  Theming
% - - - - - - - - - - - - - - - - - - - - - - - - - - - - - - - - - - - - - - -
%
% going for a simple, pastel look here
%

\usetheme{default}                       % default beamer theme (clean, empty)
\beamertemplatenavigationsymbolsempty    % remove nav symbols

\definecolor{fore}{RGB}{249,242,215}     % foreground color, off-white
\definecolor{back}{RGB}{51,51,51}        % background color, gray
\definecolor{title}{RGB}{96,148,219}     % title color, pastel blue
\definecolor{keywords}{RGB}{255,0,90}    % code keyword color, pastel pink
\definecolor{comments}{RGB}{0,179,113}   % code comment color, pastel green
\definecolor{item}{RGB}{96,148,219}      % title color, pastel blue

\setbeamercolor{titlelike}{fg=title}          % titles use title color
\setbeamercolor{normal text}{fg=fore,bg=back} % text is fore on back
\setbeamercolor{block title}{fg=comments}     % block titles are comment color
\setbeamercolor{section in toc}{fg=comments}  % toc uses title color
\setbeamercolor{item}{fg=item}                % itemize symbol color


% - - - - - - - - - - - - - - - - - - - - - - - - - - - - - - - - - - - - - - -
%  code listing
% - - - - - - - - - - - - - - - - - - - - - - - - - - - - - - - - - - - - - - -
%
% beamer and listings don't play super well together
% best to define code *outside* of a frame and then call it, like so:
%
% \defverbatim[colored]\code{           % define code out of frame
% \begin{lstlisting}
%  -- code goes here --
% \end{lstlisting}
% } % end of stored code
%
% \begin{frame}
% \frametitle{title}
% \code % call code in frame
% \end{frame}
%

\lstset{                                % listings settings
	language=[LaTeX]TeX,                % LaTeX by default
	upquote=false,                      % do NOT use "
	tabsize=4,                          % tabs are 4 spaces
	basicstyle=\ttfamily,               % code is typewriter-style
	keywordstyle=\color{keywords},      % keywords use keyword color
	commentstyle=\color{comments}\emph, % comments use comment color
}

% - - - - - - - - - - - - - - - - - - - - - - - - - - - - - - - - - - - - - - -
%  section slide command
% - - - - - - - - - - - - - - - - - - - - - - - - - - - - - - - - - - - - - - -
%
% new command to draw a simple line.  it will be as long as available,
% and 1pt wide
%

\newcommand{\srule}{
	\rule{\textwidth}{1pt}\\
}

% - - - - - - - - - - - - - - - - - - - - - - - - - - - - - - - - - - - - - - -
%  custom slide environment
% - - - - - - - - - - - - - - - - - - - - - - - - - - - - - - - - - - - - - - -
%
% essentially frame, but automatically adds section and possibly subsection as
% title.  also increases default font size
%
% method to detect subsection presence is kind of hacky:
% find it's width. if it has any width, it exists.
%

% variable to hold subsection's width
\newlength{\subsecwidth}

% slide environment - frame, plus automatic title
\newenvironment{slide}{
	\begin{frame} % frame
	\settowidth{\subsecwidth}{\insertsubsection} % find subsection width
	\ifthenelse{\dimtest{\subsecwidth}{<}{1pt}}{ % no subsection
		\frametitle{\insertsection\\             % insert *just* section
		\vspace{-1ex}                            % move next line up a bit
		\color{fore}\srule                       % pretty line
		\par                                     % remove excess spacing
		\vspace{-3ex}                            % remove excess spacing
		}
	}{                                           % subsection exists
		\frametitle{\insertsection\ -- \insertsubsection\\ % sec - subsec
		\vspace{-1ex}                            % move next line up a bit
		\color{fore}\srule                       % pretty line
		\par                                     % remove excess spacing
		\vspace{-3ex}                            % remove excess spacing
		}
	}
	\Large                                       % make font in slide Large
}{
	\end{frame}
}

% - - - - - - - - - - - - - - - - - - - - - - - - - - - - - - - - - - - - - - -
%  code result environment
% - - - - - - - - - - - - - - - - - - - - - - - - - - - - - - - - - - - - - - -
%
% code is often paired with there result, but has to be entered outside of
% frame.  use code from outside frame (\code), paired with result
%

\newenvironment{coderesult}{
	\begin{block}{Code}      % block, called Code
		\code                % print \code
	\end{block}
	\begin{block}{Result}    % block, called Result
}{                           % result argument is here
	\end{block}
}

% - - - - - - - - - - - - - - - - - - - - - - - - - - - - - - - - - - - - - - -
%  section slide command
% - - - - - - - - - - - - - - - - - - - - - - - - - - - - - - - - - - - - - - -
%
% simple command to both set the section and display a lone frame indicating
% the new section
%

\newcommand{\titleslide}[1]{
	\section{#1}             % set the section based on argument
	\begin{slide}
		\begin{center}
			\color{comments}
			\Huge            % Huge font size
			#1               % print new section's title
		\end{center}
	\end{slide}
}

% - - - - - - - - - - - - - - - - - - - - - - - - - - - - - - - - - - - - - - -
%  formatting commands
% - - - - - - - - - - - - - - - - - - - - - - - - - - - - - - - - - - - - - - -

\newcommand{\mediatitle}[1]{\textit{#1}}  % media titles should be italicized
\newcommand{\forlang}[1]{\textit{#1}}     % foreign languages should be ital

% - - - - - - - - - - - - - - - - - - - - - - - - - - - - - - - - - - - - - - -
%  formatting commands
% - - - - - - - - - - - - - - - - - - - - - - - - - - - - - - - - - - - - - - -

\renewcommand{\thefootnote}{\fnsymbol{footnote}} % fancy symbols for footnotes

% - - - - - - - - - - - - - - - - - - - - - - - - - - - - - - - - - - - - - - -
%  title block
% - - - - - - - - - - - - - - - - - - - - - - - - - - - - - - - - - - - - - - -

\title{Introduction to Weird Functional Languages with Haskell Edition}    % title
\author{
	Alex ``saterus'' Burkhart\\          % author
	The Ohio State University\\         % university
	Open Source Club                    % club
}
\date{2011-05-19}                       % date

% =============================================================================
%  actual document begins here
% =============================================================================

\begin{document}                        % settings end, content begins

\begin{frame}                           % title slide
	\srule                              % pretty line
	\titlepage                          % title page (title, author, date)
	\srule                              % pretty line
\end{frame}

\begin{frame}                           % Table of contents slide
	\begin{center}
		\srule                          % pretty line
		\vspace{1ex}
		\color{title} \inserttitle\\\color{fore} Table of Contents
		\srule                          % pretty line
	\end{center}
	\begin{columns}                     % break into two columns
		\begin{column}{.5\textwidth}    % first column, 1/2 page width
			\tableofcontents[sections={1-3}] % first three sections of ToC
		\end{column}
		\begin{column}{.5\textwidth}    % second column, 1/2 page width
			\tableofcontents[sections={4-6}] % last three sections of ToC
		\end{column}
	\end{columns}
\end{frame}

% - - - - - - - - - - - - - - - - - - - - - - - - - - - - - - - - - - - - - - -
%  Intro to FP
% - - - - - - - - - - - - - - - - - - - - - - - - - - - - - - - - - - - - - - -

\titleslide{Functional Programming}

\subsection{The Functional Paradigm}
\begin{slide}
  \begin{itemize}
    \item Declarative
    \item First Class Functions
    \item Pure Functions
    \item Immutibility
    \item Parallelism
  \end{itemize}
\end{slide}

\begin{slide}
  Declarative
  \begin{itemize}
    \item ``What'' instead of ``How''
    \item Smart Compilers
    \item Safety and Correctness
  \end{itemize}
\end{slide}

\begin{slide}
  First Class Functions
  \begin{itemize}
    \item Functions as Data Structures
    \item Use Functions as Arguments to Other Functions
    \item Abstract Common Patterns
  \end{itemize}
\end{slide}

\begin{slide}
  Pure Functions
  \begin{itemize}
    \item No I/O or Modification of State
    \item Consistent, Predictible Results
    \item Safety
    \item Optimization
  \end{itemize}
\end{slide}

\begin{slide}
  Immutability
  \begin{itemize}
    \item Can't Change Data
    \item Immutable Structures can be Shared
    \item Reusing List Links
  \end{itemize}
\end{slide}

\begin{slide}
  Parallelism
  \begin{itemize}
    \item Locks Unnecessary
    \item Order of Execution negotiated by Compiler
  \end{itemize}
\end{slide}

\begin{slide}
  Extra Stuff
  \begin{itemize}
    \item Code Generation with Lisp
    \item Type Safety with Haskell and OCaml
    \item Massive Paralellism with Erlang
    \item Haskell, Erlang, OCaml are all \textit{fast}
  \end{itemize}
\end{slide}


\titleslide{Haskell}

\subsection{Basic Syntax}

\begin{slide}
  Basic Data Structures
  \begin{itemize}
    \item Bool, Char, Numbers, List, String
    \item True and False
    \item 'a' through 'z' and more
    \item Ratios and Arbitrary Sized
    \item \code [1, 2, 3]
  \end{itemize}
\end{slide}

\begin{slide}
  Function Signatures
  \begin{itemize}
    \item 12 :: Int
    \item \code [1, 2, 3] :: [Integer]
    \item sum :: [Integer] -> Integer
    \item \code ['f', 'o', 'o'] :: String
  \end{itemize}
\end{slide}

\begin{slide}
  Algebraic Data Types
  \begin{itemize}
    \item data Bool = True | False
    \item data Int = -2147483648 | -2147483647 | ... | -1 | 0 | 1 | 2 | ... | 2147483647
    \item data Color = Red | Green | Blue
    \item data TrafficLight = Light String String Color
    \item data Maybe a = Nothing | Just a
  \end{itemize}
\end{slide}

\begin{slide}
  Function Definitions
  \begin{itemize}
    \item
      \code
      even :: Integer -> Integer -> Bool\\
      even x = (mod x 2) == 0
    \item
      odd :: Integer -> Integer -> Bool\\
      odd x = not (even x)
    \item
      doubleMe x = x + x
  \end{itemize}
\end{slide}

\subsection{Typeclasses}
\begin{slide}
  The Problem
  \begin{itemize}
    \item Function Scope
    \item Equality for the Color type\\
      \code
      (==) :: Color -> Color -> Bool\\
      (==) colorA colorB = ...
  \end{itemize}
\end{slide}

\begin{slide}
  The Solution
  \begin{itemize}
    \item Ad-hoc polymorphic interfaces
    \item
      \code
      class Eq a where\\
      (==) :: a -> a -> Bool\\
      (/=) :: a -> a -> Bool
    \item
      \code
      instance Eq Color where\\
        (==) :: Color -> Color -> Bool\\
        (==) a b = ...

        (/=) :: Color -> Color -> Bool\\
        (/=) a b = not (a == b)
  \end{itemize}
\end{slide}

\begin{slide}
  The Win
  \begin{itemize}
    \item Typeclass Deriving
    \item
      \code
      data Color = Red | Green | Blue\\
      deriving (Eq, Show, Read)

  \end{itemize}
\end{slide}






\begin{slide}
	\mediatitle{The Art of Computer Programming} (First edition)
	\begin{itemize}
		\item Published in 1969
		\item Used ``Monotype technology'' for typesetting
	\end{itemize}
\end{slide}

\begin{slide}
	\mediatitle{The Art of Computer Programming} (Second edition)
	\begin{itemize}
		\item Published in 1976
		\item ``Monotype'' no longer available
		\item Original fonts no longer available
		\item Knuth disapproved of gallery proofs
	\end{itemize}
\end{slide}

\begin{slide}
	``I'll go build my \textbf{own} theme park!  With blackjack!
	And hookers!''\\
	\bigskip      % some vertical space
	\hspace{1em}  % small indentation
	--- Bender, \mediatitle{Futurama}
\end{slide}

\subsection{\TeX} % - - - TeX specifics - - -
\begin{slide}
	\TeX~\textipa{/'tEx/} or \textipa{/'tEk/} was developed for over a decade,
	finalized in 1989 \label{ipa2}
	\begin{itemize}
		\item All changes since are bug fixes, no new features
		\item Bug fix version approaches $\pi$
		\begin{itemize}
			\item Current version is 3.1415926
		\end{itemize}
		\item Also finalized METAFONT in same year
		\begin{itemize}
			\item Version approaches $e$
		\end{itemize}
	\end{itemize}
\end{slide}

\begin{slide}
	\TeX\ is a Turing-complete language
	\begin{itemize}
		\item If-then-else
		\item Variables
		\item Loops
	\end{itemize}
\end{slide}

\begin{slide}
	\TeX\ uses a lot of neat algorithms
	\begin{itemize}
		\item Mathematical spacing
		\item Justification
		\item Hyphenation/line breaking
		\begin{itemize}
			\item Adopted by Adobe InDesign
			\item Adopted by GNU fmt
		\end{itemize}
	\end{itemize}
\end{slide}

\subsection{\TeX\ Macro Packages} % - - - TeX Macro Packages - - -
\begin{slide}
	\TeX\ is very low level, not practical by itself
	\begin{itemize}
		\item Knuth created a macro package over \TeX\ called Plain~\TeX
	\end{itemize}
\end{slide}

\begin{slide}
	AMS-\TeX
	\begin{itemize}
		\item Created by American Mathematical Society
		\item Made math formatting easier
		\item Largely abandoned for \LaTeX\ with AMS packages
	\end{itemize}
\end{slide}

\begin{slide}
	\LaTeX\ \textipa{/'le\;ItEk/}, \textipa{/'le\;ItEx/}, \textipa{/'la:tEx/},
	\textipa{/'la:tEk/} \label{ipa3}
	\begin{itemize}
		\item Created by Leslie Lamport
		\item Author focus on content, let \LaTeX\ worry about presentation
		\item By far the most popular flavor of \TeX\ today
		\item Development largely stalled
		\begin{itemize}
			\item Latest stable is \LaTeXe
			\item \LaTeX3 has been in development for longer than Duke Nukem
			Forever
		\end{itemize}
	\end{itemize}
\end{slide}

\begin{slide}
	Con\TeX t
	\begin{itemize}
		\item created by Hans Hagen
		\item easy and consistent access to advanced typographical control
		\item still in active development
		\begin{itemize}
			\item latest release was six months ago
			\item moved to git in 2009
		\end{itemize}
	\end{itemize}
\end{slide}

% - - - - - - - - - - - - - - - - - - - - - - - - - - - - - - - - - - - - - - -
%  Is LaTeX right for you?
% - - - - - - - - - - - - - - - - - - - - - - - - - - - - - - - - - - - - - - -

\titleslide{Is \LaTeX\ right for you?}

\subsection{The Test} % - - - The Test - - -

\begin{slide}
	\begin{center}
		Which is the proper lowercase delta?\\
		\vspace{1em}
%		\includegraphics{del1.png}
		\hspace{2cm}
%		\includegraphics{del2.png}\\
	\end{center}
\end{slide}

\begin{slide}
	``I don't know and I don't care.''
	\begin{itemize}
		\item<2-> \LaTeX\ probably isn't for you.  Stick with your word
		processor.
	\end{itemize}
\end{slide}

\begin{slide}
	``I don't care what the next slide says, \textit{that} one is correct.''
	\begin{itemize}
		\item<2-> \LaTeX\ probably isn't for you, but Plain~\TeX\ or Con\TeX t
		may be a good fit.
	\end{itemize}
\end{slide}

\begin{slide}
	``I don't know but I'd like to use whichever is correct.''
	\begin{itemize}
		\item<2-> \LaTeX\ may be worth your time.
	\end{itemize}
\end{slide}

\subsection{Other \LaTeX\ strengths} % - - - Other LaTeX strengths- - -

\begin{slide}
	\LaTeX\ typesets mathematics very well
	\begin{align*}
		|x|=
		\begin{cases}
			x & \text{for } x\geq 0\\
			-x & \text{for } x < 0
		\end{cases}
	\end{align*}
	\begin{align*}
		\begin{bmatrix}
			3 & 1\\
			-2 & 5
		\end{bmatrix}
		\begin{bmatrix}
			6 & 2\\
			1 & 9
		\end{bmatrix}
		=
		\begin{bmatrix}
			19 & 15\\
			-7 & 41
		\end{bmatrix}
	\end{align*}
	\begin{align*}
		\frac{
			2x+xy+y+2
		}{
			x+1
		}
		=
		\frac{
			\cancel{(x+1)}(y+2)
		}{
			\cancel{x+1}
		}
		= \boxed{y+2}
	\end{align*}
\end{slide}

\begin{slide}
	\small
	Given the equation
	\begin{align} \label{1stnonlin}
		\frac{\mathrm{d}y}{\mathrm{d}t}+p(t)y=q(t)
	\end{align}
	Find the integrating factor
	\begin{align} \label{mu}
		\mu(t)=e^{\int p(t)\mathrm{d}t}
	\end{align}
	And multiply the equation \eqref{1stnonlin} by the integrating factor
	\eqref{mu}
	\begin{align*}
		\mu\left[
			\frac{\mathrm{d}y}{\mathrm{d}t}+p(t)y
		\right]
		&=\mu q(t)
		\\
		\frac{\mathrm{d}}{\mathrm{d}t}\left[
			\mu y
		\right]
		&=\mu q(t) \tag{product rule}
		\\
		\mu y&=\int \mu q(t) \mathrm{d}t
		\\
		y&=
		\boxed
		{\frac{
			\int \mu q(t)\mathrm{d}t
		}{ \mu }}
	\end{align*}
\end{slide}

\begin{slide}
	\begin{itemize}
		\item Bibliography management / citations (Bib\TeX)
		\item Automation of many useful things
		\begin{itemize}
			\item Table of Contents
			\item Indices / Appendices
			\begin{itemize}
				\item List of Figures
				\item List of Definitions
			\end{itemize}
		\end{itemize}
		\item Easy access to many symbols
		\begin{itemize}
			\item International Phonetic Alphabet
			\footnote{See sections ``\nameref{ipa1}'', ``\nameref{ipa2}'',
			and ``\nameref{ipa3}''}
			\item Random math symbols ($\partial$, $\int$, $\infty$)
			\item Comprehensive list of symbols:
			\url{http://www.ctan.org/tex-archive/info/symbols/comprehensive/symbols-a4.pdf}
		\end{itemize}
		\item Great cross referencing\label{cross}
		\item Absolutely wonderful documentation
	\end{itemize}
\end{slide}

\subsection{Disclaimer} % - - - Disclaimer - - -

\begin{slide}
	Disclaimer:
	\begin{itemize}
		\item \LaTeX\ can be moody, temperamental and sometimes just plain
		weird.  While the general concept isn't hard, it can throw curve balls
		at you quite a lot.  It may not be worth the time for everyone, even if
		you gave the third answer.
		\item However, if you do learn all the weird kinks---or just don't do
		stuff that requires them---\LaTeX\ will hand you beautiful,
		professional-looking documents.
		\footnote{Along with other benefits, explained \vpageref{cross}.}
	\end{itemize}
\end{slide}

% - - - - - - - - - - - - - - - - - - - - - - - - - - - - - - - - - - - - - - -
%  LaTeX Philosophy
% - - - - - - - - - - - - - - - - - - - - - - - - - - - - - - - - - - - - - - -
\titleslide{\LaTeX\ Philosophy}

\begin{slide}
	\LaTeX\ Philosophy
	\begin{itemize}
		\item Where possible, let a typesetting professional decide visual
		presentation --- let the author focus on the content.
		\item There should be a clear separation between the logical structure
		of a document and the visual presentation of the document.
	\end{itemize}
\end{slide}

\subsection{\LaTeX\ is the Typesetter}
\begin{slide}
	How should you format a section heading?
	\begin{itemize}
		\item<2> Left-justified?  Centered?  Right-justified?
		\item<2> Italicized?  Bolded?  Slanted?  Underlined?
		\item<2> Monospaced font?  Proportional font?
		\item<2> Same size as body font?  Larger?  How much?
		\item<2> Numbered?  Place above? Place left?
		\item<2> Does the number go between parentheses?  Just a right
		parenthesis?  Maybe a dot?  A colon?
		\item<2> Roman numeral?  Alphabetic?  Arabic numeral?
		\item<2> Does the section number get reset each chapter?
		\item<2> What if I'm just writing a short article?  What if it's a longer
		report?  What if it's a full book?
	\end{itemize}
\end{slide}

\begin{slide}
	\begin{center}
		Don't worry about it!
	\end{center}
\end{slide}

\defverbatim[colored]\code{
\begin{lstlisting}
\documentclass{article}
\end{lstlisting}
\vdots
\begin{lstlisting}
\section{Practical Limitations
		of Special Relativity}
\end{lstlisting}
}
\begin{slide}
	Just tell \LaTeX\ what type of document it is, and where to put the section
	heading and what the title is.
	\bigskip
	\code
\end{slide}

\begin{slide}
	\LaTeX's formatting decisions are like \mediatitle{The Emperor's New
	Clothes.}
	\begin{itemize}
		\item If you don't like how \LaTeX\ formats things, you're an unfit,
		incompetent, stupid person.
		\item Except, ya'know, \LaTeX's output isn't a scam.  It's the real
		deal.
		\item \LaTeX's output is empirically perfect.
		\footnote{Except tables, which it does in a non-perfect
		manner by default.  Use Booktabs package to remedy.}
	\end{itemize}
	So let it decide for you.
\end{slide}

\subsection{Logical vs Formatted Markup}
\defverbatim[colored]\code{
\begin{lstlisting}
% formating foreign languages
\newcommand{\forlang}[1]{\textit{#1}}
\end{lstlisting}
\vdots
\begin{lstlisting}
What is the soup \forlang{du jour?}
\end{lstlisting}
}
\begin{slide}
	You should rarely use italics, bold, underline, etc directly.
	\begin{itemize}
		\item Rather, define a new command which describes what you're doing,
		and have \emph{that} decide the formatting.
	\end{itemize}
	\code
\end{slide}

\begin{slide}
	It's easy to change how everything using that command looks
	without search/replacing.
	\begin{itemize}
		\item Italicized book titles
		\item Later decided to italicize foreign language phrases
		\item Don't want to confuse a foreign language phrase with a book title
		\item Change book titles to be underlined rather than italicized
		\item Just change \textbackslash booktitle command
	\end{itemize}
\end{slide}

\begin{slide}
	It keeps you from unintentionally losing consistency with
	formatting.
	\begin{itemize}
		\item My Vim presentation used
		\begin{itemize}
			\item vim commands in several modes
			\item vim command names
			\item key combinations
			\item out-of-vim command-line commands
			\item urls
			\item examples
			\item emphasis
		\end{itemize}
		Hard to keep track what used what formatting in my head!
	\end{itemize}
\end{slide}

\subsection{} % - - - remove subsection - - -
\begin{slide}
None of the \LaTeX\ philosophy is required when using \LaTeX.
\begin{itemize}
	\item you can make your own section headings with \LaTeX
	\item you can bold things directly with \LaTeX
\end{itemize}
However, it's a good idea to follow the philosophy anyways.
\end{slide}

% - - - - - - - - - - - - - - - - - - - - - - - - - - - - - - - - - - - - - - -
%  LaTeX basics
% - - - - - - - - - - - - - - - - - - - - - - - - - - - - - - - - - - - - - - -
\titleslide{General \LaTeX\ Document Layout}

\begin{slide}
	\LaTeX\ documents can usually be broken up into two main sections:
	\begin{itemize}
		\item The Preamble
		\item The Body
	\end{itemize}
\end{slide}

\subsection{Preamble}
\begin{slide}
	The preamble is primarily used to set things up.
	\begin{itemize}
		\item This is where you indicate the type of document you are making
		\begin{itemize}
			\item Article, report, book, letter, beamer (presentations - what you
			see right now)
		\end{itemize}
		\item General Information
		\begin{itemize}
			\item Font, page size, margins
		\end{itemize}
		\item Packages
		\begin{itemize}
			\item Booktabs (improved tables), amsmath (new math environments)
		\end{itemize}
		\item New commands and environments
		\begin{itemize}
			\item Formatting commands, definition environments
		\end{itemize}
	\end{itemize}
\end{slide}

\subsection{The Body}
\begin{slide}
	After the preamble is usually the line \textbackslash begin\{document\}
	\begin{itemize}
		\item This marks the beginning of the actual document's body
		\item Everything between that and \textbackslash end\{document\} is
		what is outputted, based on the preamble
		\item Everything after \textbackslash end\{document\} is ignored
	\end{itemize}
\end{slide}

\defverbatim[colored]\code{
\begin{lstlisting}
\documentclass{minimal}
\begin{document}
Hello World!
\end{document}
\end{lstlisting}
}
\begin{slide}
Minimal \LaTeX\ document:
\code
\end{slide}

\defverbatim[colored]\code{
\begin{lstlisting}
% beginning is called the ``preamble''
% used to set things up
\documentclass{article}          % type of doc
\usepackage{booktabs}            % load package
\usepackage[T1]{fontenc}         % load package
\newcommand{\hf}{$\frac{1}{2}$}  % new command
\title{the best number}
\author{john doe}
\date{\today}
% actual document begins here
\begin{document}
\maketitle % use author, etc to make title
\section{my childhood}
Many people like integers, ever since
I was every young, I've prefered \hf
\end{document}
\end{lstlisting}
}
\begin{slide}
Slightly fuller \LaTeX\ document:
\code
\end{slide}


% - - - - - - - - - - - - - - - - - - - - - - - - - - - - - - - - - - - - - - -
%  LaTeX Basics
% - - - - - - - - - - - - - - - - - - - - - - - - - - - - - - - - - - - - - - -
\titleslide{\LaTeX\ Basics}

\defverbatim[colored]\code{
\begin{lstlisting}
You can't see comments % Like this part
\end{lstlisting}
}
\subsection{Comments} % - - - comments - - -
\begin{slide}
	\begin{itemize}
		\item Comments start with \%
		\item No multi-line comments by default
	\end{itemize}
	\begin{coderesult}
		You can't see comments % Like this part
	\end{coderesult}
\end{slide}

\subsection{Whitespace} % - - - whitespace - - -
\begin{slide}
	\begin{itemize}
		\item \LaTeX\ treats consecutive whitespace as a single space
		\item except two consecutive newlines which indicate a paragraph break
	\end{itemize}
\end{slide}

\defverbatim[colored]\code{
\begin{lstlisting}[showtabs=true,showspaces=true]
This        code
	will all show   		   up
on one	 line.
\end{lstlisting}
}
\begin{slide}
	\begin{coderesult}
	This        code
		will all show   		   up
	on one	 line.
	\end{coderesult}
\end{slide}

\defverbatim[colored]\code{
\begin{lstlisting}
The fact that you can ignore
newlines can   % it allows comments mid-sentence
be quite useful.
\end{lstlisting}
}
\begin{slide}
	\begin{coderesult}
		The fact that you can ignore
		newlines can   % it allows comments mid-sentence
		be quite useful.
	\end{coderesult}
\end{slide}

\defverbatim[colored]\code{
\begin{lstlisting}[showtabs=true]
	you can also
		play with indentation
	to pretty-fy code
		without changing output
\end{lstlisting}
}
\begin{slide}
	\begin{coderesult}
	you can also
		play with indentation
	to pretty-fy code
		without changing output
	\end{coderesult}
\end{slide}

\defverbatim[colored]\code{
\begin{lstlisting}
This is a short paragraph about stuff and/or
things.  I'm not exactly sure what to say.

This is  new paragraph.  Could be about other
things, like pizza.
\end{lstlisting}
}


\begin{slide}
	\begin{coderesult}
	\hspace{2em}This is a short paragraph about stuff and/or
	things.  I'm not exactly sure what to say.

	\hspace{2em}This is  new paragraph.  Could be about other
	things, like pizza.
		\end{coderesult}
\end{slide}

\subsection{Commands} % - - - commands - - -
\begin{slide}
	\begin{itemize}
		\item by default, commands start with \textbackslash
		\item two different types of commands
		\begin{itemize}
			\item multi-character
			\item single-character
		\end{itemize}
	\end{itemize}
\end{slide}

\defverbatim[colored]\code{
\begin{lstlisting}[showspaces=true]
Con\TeX t is cool
but \LaTeX is cooler
% Note lack of space after X when compiled
\end{lstlisting}
}
\begin{slide}
	multi-character commands
	\begin{itemize}
		\item \LaTeX\ reads from \textbackslash\ to first non-alphabetic
		character.
		\item If ends with whitespace, eats it.
	\end{itemize}
	\begin{block}{Code}
	\code
	\end{block}
	\begin{block}{Result}
	Con\TeX t is cool
	but \LaTeX is cooler
	\end{block}
\end{slide}

\defverbatim[colored]\code{
\begin{lstlisting}[showspaces=true]
Con\TeX t is cool
but \LaTeX\ is cooler % can't eat escaped space
\end{lstlisting}
}
\begin{slide}
	single-character commands
	\begin{itemize}
		\item Some commands are only one character.
		\item may not be alphabetic
	\begin{block}{Code}
	\code
	\end{block}
	\begin{block}{Result}
	Con\TeX t is cool
	but \LaTeX\ is cooler % can't eat escaped space
	\end{block}
	\end{itemize}
\end{slide}


\defverbatim[colored]\code{
\begin{lstlisting}
To break lines\\ in   % line breaks at \\
\LaTeX,               % individual LF is ignored
use \textbackslash\textbackslash
\end{lstlisting}
}
\begin{slide}
	to break a line, but not make a new paragraph, use
	\textbackslash\textbackslash
	\begin{block}{Code}
	\code
	\end{block}
	\begin{block}{Result}
	To break lines\\ in   % line breaks at \\
	\LaTeX,               % individual LF is ignored
	use \textbackslash\textbackslash
	\end{block}
\end{slide}

\begin{slide}
	Reserved Characters
	\begin{itemize}
		\item Many other characters are reserved but can be accessed by
		escaping (e.g.\@ \textbackslash \%)
		\begin{itemize}
			\item \#, \$, \%, \&, \_, \{, \}
		\end{itemize}
		\item others need a \{\} after their escaped version
		(e.g.\textbackslash \@\~{}\{\})
		\begin{itemize}
			\item \~{}, \^{}
		\end{itemize}
		\item still others need a command
		\begin{itemize}
			\item \textbackslash textbackslash yields a \textbackslash
		\end{itemize}
	\end{itemize}
\end{slide}

\subsection{Arguments/Curly Brackets} % - - - arguments - - -

\defverbatim[colored]\code{
\begin{lstlisting}
Sometimes you just have to be \textbf{bold.}
\end{lstlisting}
}
\begin{slide}
	Many commands have arguments.
	\begin{itemize}
		\item Mandatory arguments usually go in \{\} pairs
		\item Optional arguments usually go in [] pairs
	\end{itemize}
	\begin{coderesult}
		Sometimes you just have to be \textbf{bold.}
	\end{coderesult}
\end{slide}

\defverbatim[colored]\code{
\begin{lstlisting}
Most arguments are actually
only one \textbf token.
\end{lstlisting}
}
\begin{slide}
	Technically, \{\} combine multiple tokens into one token.
	\begin{itemize}
		\item think C-style \{\} grouping multiple lines into one.
		\item most commands with arguments are looking at the next token
	\end{itemize}
	\begin{coderesult}
		Most arguments are actually only one \textbf token.
	\end{coderesult}
\end{slide}

\defverbatim[colored]\code{
\begin{lstlisting}
Curley braces are {\tiny also used for} scoping
\end{lstlisting}
}

\begin{slide}
	Curly braces are also used for scoping
	\begin{coderesult}
	Curley braces are {\tiny also used for\/} scoping
	\end{coderesult}
\end{slide}

\subsection{Environments} % - - - environments - - -

\begin{slide}
	Environments are effectively commands that are intended to have large or
	special arguments.
	\begin{itemize}
		\item They begin with \textbackslash begin\{environment-name\}
		\item and end with \textbackslash end\{environment-name\}
		\item can be used in weird cases where normal commands wouldn't work
	\end{itemize}
\end{slide}

\defverbatim[colored]\code{
\begin{lstlisting}
\begin{center}
	This text is centered
\end{center}
\end{lstlisting}
}
\begin{slide}
	\begin{coderesult}
		\begin{center}
			This text is centered
		\end{center}
	\end{coderesult}
\end{slide}

\subsection{Modes} % - - - modes - - -

\begin{slide}
	\LaTeX\ has three modes
	\begin{itemize}
		\item paragraph mode - default, normal
		\item math mode - for typesetting mathematics
		\item LR mode - not normally important
	\end{itemize}
\end{slide}

\begin{slide}
	To get to math mode:
	\begin{itemize}
		\item in-line math is between single \$'s
		\item A few ways to get to display math:
		\begin{itemize}
			\item between \$\$'s (not recommended)
			\item between \textbackslash[ \textbackslash]
			\item within special environments (such as equation)
		\end{itemize}
	\end{itemize}
\end{slide}

\defverbatim[colored]\code{
\begin{lstlisting}
\begin{equation}
	ax^2+bx+c=0 \label{eqn:quadratic}
\end{equation}
How do I find the roots of $x$
in equation \eqref{eqn:quadratic}
\end{lstlisting}
}
\begin{slide}
	\begin{coderesult}
	\begin{equation}
		ax^2+bx+c=0 \label{eqn:quadratic}
	\end{equation}
	How do I find the roots of $x$ in equation \eqref{eqn:quadratic}
	\end{coderesult}
\end{slide}

% - - - - - - - - - - - - - - - - - - - - - - - - - - - - - - - - - - - - - - -
% Where to go to learn more
% - - - - - - - - - - - - - - - - - - - - - - - - - - - - - - - - - - - - - - -

\renewcommand{\thefootnote}{\arabic{footnote}} % numbered footnotes
\setcounter{footnote}{0}                       % reset footnote numbering
\titleslide{Where to go to Learn More}
\begin{slide}
	\begin{itemize}
		\normalsize
		\item Wikibooks
		\footnote{
			HTML:\\ \url{https://secure.wikimedia.org/wikibooks/en/wiki/LaTeX}
		}
		\footnote{
			PDF:\\ \url{http://upload.wikimedia.org/wikipedia/commons/2/2d/LaTeX.pdf}
		}
		\item The Not So Short Introduction to \LaTeXe
		\footnote{
			PDF:\\ \url{http://www.ctan.org/tex-archive/info/lshort/english/lshort.pdf}
		}
		\item \mediatitle{The \TeX book} by Donald E. Knuth (Plain \TeX)
		\footnote{
		 	\TeX:\\
			\url{http://tex.loria.fr/general/texbook.tex}
		}
		\footnote{
		 \$32.20:
			\url{http://www.amazon.com/TeXbook-Donald-Knuth/dp/0201134489}
		}
		\item \mediatitle{\LaTeX: A Document Preparation System} by Leslie Lamport
		\footnote{
		 \$29.99:
			\url{http://www.amazon.com/LaTeX-Document-Preparation-System-2nd/dp/0201529831}
		}
		\item \mediatitle{The \LaTeX\ Companion}
		\footnote{
		 \$54.56:
			\url{http://www.amazon.com/gp/product/0201362996}
		}
	\end{itemize}
\end{slide}

\end{document}
\section{\LaTeX\ gotchas}
\section{Where to go from here}

latex   - 219
tex     - 5
context - 9

Beamer user guide - 240p
pgf/tikz manual -


\definecolor{item}{RGB}{253,198,137}     % item color, pastel peach
