% =============================================================================
%  saterus's slides for a presentation on Haskell
% =============================================================================

\documentclass{beamer}                  % beamer class, good for slides
\usepackage[T1]{fontenc}                % use a modern scalable font
\usepackage{lmodern}                    % Latin Modern font
\usepackage[T1]{tipa}                   % pronunciation symbols
\usepackage{graphicx}                   % include pictures
\usepackage{xifthen}                    % \ifthenelse
\usepackage{framed}                     % framed environment
\usepackage{cancel}                     % \cancel in math
\usepackage{varioref}                   % \vpageref cross referencing
\usepackage{listings}                   % code formatting
\DeclareGraphicsExtensions{.png}

% - - - - - - - - - - - - - - - - - - - - - - - - - - - - - - - - - - - - - - -
%  Theming
% - - - - - - - - - - - - - - - - - - - - - - - - - - - - - - - - - - - - - - -
%
% going for a simple, pastel look here
%

\usetheme{default}                       % default beamer theme (clean, empty)
\beamertemplatenavigationsymbolsempty    % remove nav symbols

\definecolor{fore}{RGB}{249,242,215}     % foreground color, off-white
\definecolor{back}{RGB}{51,51,51}        % background color, gray
\definecolor{title}{RGB}{96,148,219}     % title color, pastel blue
\definecolor{keywords}{RGB}{255,0,90}    % code keyword color, pastel pink
\definecolor{comments}{RGB}{0,179,113}   % code comment color, pastel green
\definecolor{item}{RGB}{96,148,219}      % title color, pastel blue

\setbeamercolor{titlelike}{fg=title}          % titles use title color
\setbeamercolor{normal text}{fg=fore,bg=back} % text is fore on back
\setbeamercolor{block title}{fg=comments}     % block titles are comment color
\setbeamercolor{section in toc}{fg=comments}  % toc uses title color
\setbeamercolor{item}{fg=item}                % itemize symbol color


% - - - - - - - - - - - - - - - - - - - - - - - - - - - - - - - - - - - - - - -
%  code listing
% - - - - - - - - - - - - - - - - - - - - - - - - - - - - - - - - - - - - - - -
%
% beamer and listings don't play super well together
% best to define code *outside* of a frame and then call it, like so:
%
% \defverbatim[colored]\code{           % define code out of frame
% \begin{lstlisting}
%  -- code goes here --
% \end{lstlisting}
% } % end of stored code
%
% \begin{frame}
% \frametitle{title}
% \code % call code in frame
% \end{frame}
%

\lstset{                                    % listings settings
	language=Haskell,                   % LaTeX by default
	upquote=false,                      % do NOT use "
	tabsize=2,                          % tabs are 4 spaces
	basicstyle=\ttfamily,               % code is typewriter-style
	keywordstyle=\color{keywords},      % keywords use keyword color
	commentstyle=\color{comments}\emph, % comments use comment color
}

% - - - - - - - - - - - - - - - - - - - - - - - - - - - - - - - - - - - - - - -
%  section slide command
% - - - - - - - - - - - - - - - - - - - - - - - - - - - - - - - - - - - - - - -
%
% new command to draw a simple line.  it will be as long as available,
% and 1pt wide
%

\newcommand{\srule}{
	\rule{\textwidth}{1pt}\\
}

% - - - - - - - - - - - - - - - - - - - - - - - - - - - - - - - - - - - - - - -
%  custom slide environment
% - - - - - - - - - - - - - - - - - - - - - - - - - - - - - - - - - - - - - - -
%
% essentially frame, but automatically adds section and possibly subsection as
% title.  also increases default font size
%
% method to detect subsection presence is kind of hacky:
% find it's width. if it has any width, it exists.
%

% variable to hold subsection's width
\newlength{\subsecwidth}

% slide environment - frame, plus automatic title
\newenvironment{slide}{
	\begin{frame} % frame
	\settowidth{\subsecwidth}{\insertsubsection} % find subsection width
	\ifthenelse{\dimtest{\subsecwidth}{<}{1pt}}{ % no subsection
		\frametitle{\insertsection\\             % insert *just* section
		\vspace{-1ex}                            % move next line up a bit
		\color{fore}\srule                       % pretty line
		\par                                     % remove excess spacing
		\vspace{-3ex}                            % remove excess spacing
		}
	}{                                           % subsection exists
		\frametitle{\insertsection\ -- \insertsubsection\\ % sec - subsec
		\vspace{-1ex}                            % move next line up a bit
		\color{fore}\srule                       % pretty line
		\par                                     % remove excess spacing
		\vspace{-3ex}                            % remove excess spacing
		}
	}
	\Large                                       % make font in slide Large
}{
	\end{frame}
}

% - - - - - - - - - - - - - - - - - - - - - - - - - - - - - - - - - - - - - - -
%  code result environment
% - - - - - - - - - - - - - - - - - - - - - - - - - - - - - - - - - - - - - - -
%
% code is often paired with there result, but has to be entered outside of
% frame.  use code from outside frame (\code), paired with result
%

\newenvironment{coderesult}{
	\begin{block}{Code}      % block, called Code
		\code                % print \code
	\end{block}
	\begin{block}{Result}    % block, called Result
}{                           % result argument is here
	\end{block}
}

% - - - - - - - - - - - - - - - - - - - - - - - - - - - - - - - - - - - - - - -
%  section slide command
% - - - - - - - - - - - - - - - - - - - - - - - - - - - - - - - - - - - - - - -
%
% simple command to both set the section and display a lone frame indicating
% the new section
%

\newcommand{\titleslide}[1]{
	\section{#1}             % set the section based on argument
	\begin{slide}
		\begin{center}
			\color{comments}
			\Huge            % Huge font size
			#1               % print new section's title
		\end{center}
	\end{slide}
}

% - - - - - - - - - - - - - - - - - - - - - - - - - - - - - - - - - - - - - - -
%  formatting commands
% - - - - - - - - - - - - - - - - - - - - - - - - - - - - - - - - - - - - - - -

\newcommand{\mediatitle}[1]{\textit{#1}}  % media titles should be italicized
\newcommand{\forlang}[1]{\textit{#1}}     % foreign languages should be ital

% - - - - - - - - - - - - - - - - - - - - - - - - - - - - - - - - - - - - - - -
%  formatting commands
% - - - - - - - - - - - - - - - - - - - - - - - - - - - - - - - - - - - - - - -

\renewcommand{\thefootnote}{\fnsymbol{footnote}} % fancy symbols for footnotes

% - - - - - - - - - - - - - - - - - - - - - - - - - - - - - - - - - - - - - - -
%  title block
% - - - - - - - - - - - - - - - - - - - - - - - - - - - - - - - - - - - - - - -

\title{Introduction to Weird Functional Languages with Haskell}    % title
\author{
	Alex Burkhart\\          % author
	The Ohio State University\\          % university
}

% =============================================================================
%  actual document begins here
% =============================================================================

\begin{document}                        % settings end, content begins

\begin{frame}                           % title slide
	\srule                              % pretty line
	\titlepage                          % title page (title, author, date)
	\srule                              % pretty line
\end{frame}

\begin{frame}                           % Table of contents slide
	\begin{center}
		\srule                          % pretty line
		\vspace{1ex}
		\color{title} \inserttitle\\\color{fore} Table of Contents
		\srule                          % pretty line
	\end{center}
	\begin{columns}                     % break into two columns
		\begin{column}{.5\textwidth}    % first column, 1/2 page width
			\tableofcontents[sections={1-3}] % first three sections of ToC
		\end{column}
		\begin{column}{.5\textwidth}    % second column, 1/2 page width
			\tableofcontents[sections={4-6}] % last three sections of ToC
		\end{column}
	\end{columns}
\end{frame}

% - - - - - - - - - - - - - - - - - - - - - - - - - - - - - - - - - - - - - - -
%  Intro to FP
% - - - - - - - - - - - - - - - - - - - - - - - - - - - - - - - - - - - - - - -

\titleslide{Functional Programming}

\subsection{The Functional Discipline}

\begin{slide}
  Discipline: Restrictions with Benefits
  \begin{itemize}
    \item Eliminate GOTOs
    \item Eliminate Global Variables
    \item Eliminate the ``Useful and Flexible, but Unpredictable''
  \end{itemize}
\end{slide}

\subsection{The Functional Paradigm}

\begin{slide}
  \begin{itemize}
    \item Declarative
    \item First Class Functions
    \item Pure Functions
    \item Immutibility
    \item Parallelism
  \end{itemize}
\end{slide}

\begin{slide}
  Declarative
  \begin{itemize}
    \item ``What'' instead of ``How''
    \item Transformations of Data
    \item Smart Compilers
    \item Safety and Correctness
  \end{itemize}
\end{slide}

\begin{slide}
  First Class Functions
  \begin{itemize}
    \item Functions as Data Structures
    \item Use Functions as Arguments to Other Functions
    \item Abstract Common Patterns
  \end{itemize}
\end{slide}

\begin{slide}
  Pure Functions
  \begin{itemize}
    \item Consistent, Predictible Results
    \item No I/O or Modification of State
    \item Safety
    \item Optimization
    \item Testing
  \end{itemize}
\end{slide}

\begin{slide}
  \textbf{Referential Transparency}:
  Every occurance of a function f can be replaced with its return value without affecting the observable result of the program.
  \begin{itemize}
    \item public int foo(Thing x) \{\\
  ~~int z = bar(x);\\
  ~~\textbackslash\textbackslash do some stuff\\
  ~~launchMissiles(x);\\
  ~~\textbackslash\textbackslash do some more stuff...\\
  ~~return z + 1;\\
  \}
  \end{itemize}
\end{slide}

\begin{slide}
  Immutability
  \begin{itemize}
    \item Can't Change Data
    \item Immutable Structures can be Shared
    \item e.g. Reusing Cons Cells / List Links
  \end{itemize}
\end{slide}

\begin{slide}
  Parallelism
  \begin{itemize}
    \item Locks Unnecessary (less necessary)
    \item Order of Execution negotiated by Compiler
  \end{itemize}
\end{slide}

\begin{slide}
  Extra Stuff
  \begin{itemize}
    \item Code Generation with Lisp
    \item Type Safety with Haskell and OCaml
    \item Massive Paralellism with Erlang
    \item Haskell, Erlang, OCaml are all \textit{fast}
  \end{itemize}
\end{slide}


\titleslide{Haskell}

\subsection{Basic Syntax}

\begin{slide}
  Basic Data Structures (nothing is an object)
  \begin{itemize}
    \item Booleans: True and False
    \item Characters: 'a' through 'z' and all your favorites
    \item Numbers: Classics with Ratios \& Arbitrary Size/Precision Numbers
    \item Functions: toUpper, sum, (+), (:)
    \item{}{} Lists [1, 2, 3] == 1 : 2 : 3 : []
    \item{}{} Ranges: [1..100], ['a'..'z'], [10,9.5..0], ['z','y'..'a']
    \item Tuples: (1, 3), ('T', [1,2,3]), (4, 'O', 4)
  \end{itemize}
\end{slide}

\begin{slide}
  Functions Definitions
  \begin{itemize}
    \item twelve = 12
    \item doubleMe x = x + x
    \item even x = ((mod x 2) == 0)
    \item odd x = (not (even x))
    \item arbitrary f lst x = ((f x) : lst)
  \end{itemize}
\end{slide}

\begin{slide}
  Algebraic Data Types
  \begin{itemize}
    \item data Bool = True | False
    \item data Color = Red | Green | Blue
    \item data TrafficLight = Light [Char] [Char] Color
  \end{itemize}
\end{slide}

\begin{slide}
  Type Parameters
  \begin{itemize}
    \item data Maybe a = Nothing | Just a
    \item data Either a b = Left a | Right b
  \end{itemize}
\end{slide}

\subsection{With Types}

\begin{slide}
  Haskell's Type System
  \begin{itemize}
    \item \textbf{Strong} \& Weak Typing
    \item \textbf{Static} \& Dynamic Typing
    \item Typecheck instead of ``Run \& Pray''
    \item Type Inference \& GHCi
  \end{itemize}
\end{slide}

\begin{slide}
  Function Signatures
  \begin{itemize}
    \item 12 :: Int
    \item{}{} [1, 2, 3] :: [Int]
    \item{}{} ['f', 'o', 'o'] :: [Char]
    \item ``foo'' :: [Char]
    \item (4, 'O', 4) :: (Int, Char, Int)
    \item sum :: [Int] -> Int
  \end{itemize}
\end{slide}

\begin{slide}
  Quiz:\\
  \begin{itemize}
    \item ? :: Int -> Int -> Int
  \end{itemize}
\end{slide}

\begin{slide}
  Quiz:\\
  \begin{itemize}
    \item ? :: Int -> Int -> Int
    \item ? :: [Int] -> [Int]
  \end{itemize}
\end{slide}

\begin{slide}
  Function Definitions
  \begin{itemize}
    \item
      doubleMe :: Int -> Int\\
      doubleMe x = x + x
    \item
      even :: Int -> Bool\\
      even x = (mod x 2) == 0
    \item
      odd x = not (even x)
  \end{itemize}
\end{slide}

\begin{slide}
  Value Constructors
  \begin{itemize}
    \item data Bool = True | False
    \item The parts after the = are \textit{Value Constructors}.\\
      e.g. True and False
  \end{itemize}
\end{slide}

\begin{slide}
  Type Constructors
  \begin{itemize}
    \item data Maybe a = Nothing | Just a
    \item The parts before the = are \texit{Type Constructors}.\\
      e.g. Maybe Int
  \end{itemize}
\end{slide}

\subsection{Typeclasses}

\begin{slide}
  Typeclasses
  \begin{itemize}
    \item The Problem: Function Scope
    \item Equality for the Color type\\

      (==) :: Color -> Color -> Bool\\
      (==) colorA colorB = ...
  \end{itemize}
\end{slide}

\begin{slide}
  Typeclasses
  \begin{itemize}
    \item The Solution: Ad-Hoc Polymorphic Interfaces
    \item
      class Foo var where\\
      ~~bar x :: a -> b\\
      ~~baz x y :: a -> b -> c\\
      ~~reFoo f :: var -> var
  \end{itemize}
\end{slide}

\begin{slide}
  \begin{itemize}
  \item \textit{(Somewhere in the core libraries...)}
    \item
      class Eq a where\\
      ~~(==) :: a -> a -> Bool\\
      ~~(/=) :: a -> a -> Bool
    \item \textit{(Later on, in our file...)}
    \item
      instance Eq Color where\\
      ~~(==) :: Color -> Color -> Bool\\
      ~~(==) a b = ...

      ~~(/=) :: Color -> Color -> Bool\\
      ~~(/=) a b = not (a == b)
  \end{itemize}
\end{slide}

\begin{slide}
  Function Signatures Updated
  \begin{itemize}
    \item 12 :: Num a => a
    \item{} [1,2,3] :: Num a => [a]
    \item even :: Integral a => a -> a -> Bool
    \item sort :: Ord a => [a] -> [a]
  \end{itemize}
\end{slide}

\begin{slide}
  Bonus!
  \begin{itemize}
    \item Typeclass Deriving
    \item
      data Color = Red | Green | Blue\\
      ~~deriving (Eq, Show, Read)
    \item Compiler can automatically derive instances of Read, Show,
      Bounded, Enum, Eq, and Ord.
    \item Libraries add additional derivations.
  \end{itemize}
\end{slide}

\subsection{Pattern Matching}

\begin{slide}
  isRed :: Color -> Bool\\
  isRed c = if c == Red\\
  ~~~~~~~~~~~~~~then True\\
  ~~~~~~~~~~~~~~else False\\
  \mbox{}\\
  equals :: Color -> Color -> Bool\\
  equals x y = if x == Red \&\& y == Red...
\end{slide}

\begin{slide}
  Avoid manual equality comparisons
  \begin{itemize}
    \item data Color = Red | Green | Blue
    \item Implement Show ~~~\textit{(toString)}
    \item Implement (==)
  \end{itemize}
\end{slide}

\begin{slide}
  Pattern Matching: Pick data structures apart based on their constructors
  \begin{itemize}
    \item data Color = Red | Green | Blue
      \item
        show :: Color -> String\\
        show Red = ``Red''\\
        show Green = ``Green''\\
        show Blue = ``Blue''
      \item No ``else'' case needed
  \end{itemize}
\end{slide}

\begin{slide}
  \begin{itemize}
    \item data Color = Red | Green | Blue
    \item
        (==) :: Color -> Color -> Bool\\
        (==) Red Red     = True\\
        (==) Green Green = True\\
        (==) Blue Blue   = True\\
        (==) \_ \_         = False
  \end{itemize}
\end{slide}

\begin{slide}
  data Maybe a = Nothing | Just a
  \begin{itemize}
    \item val1 = Just 73
    \item val2 = Nothing
    \item possiblyDouble :: Maybe Int -> Maybe Int\\
      possiblyDouble x = ...?
  \end{itemize}
\end{slide}

\begin{slide}
  data Maybe a = Nothing | Just a
  \begin{itemize}
    \item val1 = Just 73
    \item val2 = Nothing
    \item possiblyDouble :: Maybe Int -> Maybe Int\\
      possiblyDouble Nothing = Nothing\\
      possiblyDouble (Just x) = Just (x + x)
  \end{itemize}
\end{slide}

\begin{slide}
  Matching inside Lists
  \begin{itemize}
    \item myList = [1,2,3,4,5]
    \item sum :: [Int] -> Int -> Int\\
          sum [] acc     = acc\\
          sum (x:xs) acc = sum xs (acc + x)
  \end{itemize}
\end{slide}

\begin{slide}
  Matching with Case Statements
  \begin{itemize}
    \item val1 = Just 73
    \item val2 = Nothing
    \item possiblyDouble :: Maybe Int -> Maybe Int\\
      possibleDouble x = case x of\\
      ~~~~Nothing -> Nothing\\
      ~~~~(Just x) -> Just (x + x)
  \end{itemize}
\end{slide}

\begin{slide}
  Guards
  \begin{itemize}
    \item val1 = 5
    \item val2 = 9001
    \item doubleSmall :: Int -> Int\\
      doubleSmall x\\
      | x <= 9000 = x + x\\
      | otherwise = x
  \end{itemize}
\end{slide}

\titleslide{Think Functionally}

\subsection{Recursion with Lists}

\begin{slide}
  \begin{itemize}
    \item myList = [1,2,3,4,5]
    \item sum :: [Int] -> Int -> Int\\
          sum [] acc     = acc\\
          sum (x:xs) acc = sum xs (acc + x)
  \end{itemize}
\end{slide}

\subsection{Functions as Arguments}

\begin{slide}
  \begin{itemize}
    \item map :: (a -> b) -> [a] -> [b]
    \item myList = [1,2,3,4,5]
    \item double :: Int -> Int
  \end{itemize}
\end{slide}

\begin{slide}
  \begin{itemize}
    \item example = ``things''
    \item toUpper :: Char -> Char
    \item map :: (a -> b) -> [a] -> [b]
  \end{itemize}
\end{slide}

\begin{slide}
  \begin{itemize}
    map :: (a -> b) -> [a] -> [b]
    map _ []     = []
    map f (x:xs) = f x : map f xs
  \end{itemize}
\end{slide}

\begin{slide}
  \begin{itemize}
    \item myList = [1,2,3,4,5]
    \item foldl :: (a -> b -> a) -> a -> [b] -> a
    \item sum = ?
  \end{itemize}
\end{slide}

\subsection{Higher Order Functions}

\begin{slide}
  \begin{itemize}
    \item map :: (a -> b) -> [a] -> [b]
    \item filter :: (a -> Bool) -> [a] -> [a]
    \item zipWith :: (a -> b -> c) -> [a] -> [b] -> [c]
    \item foldl :: (a -> b -> a) -> a -> [b] -> a
    \item foldl1 :: (a -> a -> a) -> [a] -> a
    \item foldr :: (a -> b -> b) -> b -> [a] -> b
    \item foldr1 :: (a -> a -> a) -> [a] -> a
    \item any :: (a -> Bool) -> [a] -> Bool
    \item all :: (a -> Bool) -> [a] -> Bool
  \end{itemize}
\end{slide}

\subsection{Lambdas}

\begin{slide}
  \begin{itemize}
    \item myList = [1,2,3,4,5]
    \item triple = map (\textbackslash x -> x + x + x) myList
    \item divisThree = filter (\textbackslash x -> (mod x 3) == 0) myList
  \end{itemize}
\end{slide}

\subsection{Currying}

\begin{slide}
  \begin{itemize}
    \item
      ghci> :t (+)\\
      (+) :: Num a => a -> a -> a\\
      ghci> :t (+ 1)\\
      (+ 1) :: Num a => a -> a\\
    \item
      applyTwice :: (a -> a) -> a -> a\\
      applyTwice f x = f (f x)
  \end{itemize}
\end{slide}

\subsection{Flip}
\begin{slide}
  \begin{itemize}
    \item flip :: (a -> b -> c) -> b -> a -> c
    \item flip f x y = f y x
  \end{itemize}
\end{slide}

\begin{slide}
  \begin{itemize}
    \item myDiv :: Fractional a => a -> a -> a
    \item myDiv x y = x / y
    \item myDiv 3 12\\
      => ?
    \item myDiv 36 12\\
      => ?
  \end{itemize}
\end{slide}

\begin{slide}
  \begin{itemize}
    \item flip :: (a -> b -> c) -> b -> a -> c
    \item myDiv :: Fractional a => a -> a -> a
    \item ghci> :t flip myDiv\\
      => ?
    \item flip myDiv 3 12\\
      => ?
    \item flip myDiv 36 12\\
      => ?
  \end{itemize}
\end{slide}

\subsection{Function Composition}

\begin{slide}
  \begin{itemize}
    \item \((f \circ g)(x) \equiv f(g(x))\) -- math, not Haskell
    \item (.) :: (b -> c) -> (a -> b) -> a -> c
    \item f . g = \textbackslash x -> f (g x)
    \item minimum = head . sort
    \item{}{}
      map (\textbackslash x -> negate (abs x)) [5,-3,-6,7,-3,2,-19]\\
      => [-5,-3,-6,-7,-3,-2,-19]
    \item{}{}
      map (negate . abs) [5,-3,-6,7,-3,2,-19]\\
      => [-5,-3,-6,-7,-3,-2,-19]
  \end{itemize}
\end{slide}

\subsection{Laziness}

\begin{slide}
  Delayed Computation Until Necessary
  \begin{itemize}
    \item Programs are Transformations of Data
    \item Infinite Data Structures
    \item Interesting Compiler Optimizations
    \item ``undefined''
    \item Circular Structures
    \item Unique to Haskell
  \end{itemize}
\end{slide}

\begin{slide}
  Infinite Data Structures
  \begin{itemize}
    \item Haskell Ranges
    \item  [1..] :: (Num t) => [t]
    \item take 5 [1..]
    \item length [1..] -- Bad Idea
    \item fib = 0 : 1 : zipWith (+) fib (tail fib)
  \end{itemize}
\end{slide}

\begin{slide}
  Short cut Fusion
  \begin{itemize}
    \item map f (map g someList)\\
      => map (f . g) someList
  \end{itemize}
\end{slide}

\begin{slide}
  undefined
  \begin{itemize}
    \item Valid as long as unevaluated
    \item Stub out function signatures
  \end{itemize}
\end{slide}

\begin{slide}
  Circular Structures
  \begin{itemize}
    \item No mutable references
    \item data Foo = Bar a Foo
    \item
      circularFoo :: Foo Int\\
      circularFoo = x\\
        ~~~~where x = Bar 1 y\\
        ~~~~~~~~~~~~y = Bar 2 x
      \item Sci-Fi-Explanation: ``You can borrow things from the future as long as you don't try to change them''
  \end{itemize}
\end{slide}

\begin{slide}
  Circular List
  \begin{itemize}
    \item cycle ``hey''
    \item \includegraphics[width=\linewidth]{circular_list}
  \end{itemize}
\end{slide}

\begin{slide}
  Reusing Cons Cells / List Links
  \begin{itemize}
    \item h = ``hey''\\
    \item (h, ('k':tail h))\\
    \item \includegraphics[width=\linewidth]{cons_reuse}
  \end{itemize}
\end{slide}


\titleslide{Custom Data Structures}
\subsection{List}

\begin{slide}
  \begin{itemize}
    \item data List a = Nil | Cons a (List a)
    \item
      head :: List a -> Maybe a\\
      head Nil        = Nothing\\
      head (Cons x \_) = Just x
    \item
      map :: (a -> b) -> List a -> List b\\
      map \_ Nil        = Nil\\
      map f (Cons x xs) = Cons (f x) (map f xs)
  \end{itemize}
\end{slide}

\subsection{Binary Tree}
\begin{slide}
  \begin{itemize}
    \item data Tree a = Empty | Node a (Tree a) (Tree a)
    \item testTree = TODO
    \item
      tMap :: (a -> b) -> Tree a -> Tree b\\
      tMap f (Leaf x)   = Leaf (f x)\\
      tMap f (Node l r) = Node (tMap f l) (tMap f r)
  \end{itemize}
\end{slide}

\begin{slide}
  \begin{itemize}
    \item
      tFilter :: (a -> Bool) -> Tree a -> [a]\\
      tFilter f (Leaf x)\\
      ~~~~| f x = [x]\\
      ~~~~| otherwise = []\\
      tFilter f (Node l r) = (tFilter f l) ++ (tFilter f l)
  \end{itemize}
\end{slide}

\begin{slide}
  \begin{itemize}
    \item
      tFoldDf :: (a -> b -> a) -> a -> Tree b -> a\\
      tFoldDf f acc (Leaf x)   = f acc x\\
      tFoldDf f acc (Node l r) = tFoldDf f (tFoldDf f acc l) r
    \item treeMax t = tFoldDf max 0 t
    \item treeMin t = tFoldDf min 0 t
  \end{itemize}
\end{slide}

\begin{slide}
  \begin{itemize}
    \item
      member :: (Eq a) => Tree a -> a -> Maybe a\\
      member t e = tFoldDf member' Nothing t\\
      ~~where member' acc x\\
      ~~~~~~| x == e    = Just x\\
      ~~~~~~| otherwise = acc

  \end{itemize}
\end{slide}

\titleslide{Common Patterns}
\subsection{Functors}

\begin{slide}
  \begin{itemize}
    \item map :: (a -> b) -> [a] -> [b]
    \item lmap :: (a -> b) -> List a -> List b
    \item tMap :: (a -> b) -> Tree a -> Tree b
  \end{itemize}
\end{slide}

\begin{slide}
  \begin{itemize}
    \item
      class Functor a where\\
      ~~fmap :: Functor f => (a -> b) -> f a -> f b
    \item
      instance Functor [a] where\\
      ~~fmap :: (a -> b) -> [a] -> [b]\\
      ~~fmap = map
    \item
      instance Functor (Tree a) where\\
      ~~fmap :: (a -> b) -> Tree a -> Tree b\\
      ~~fmap = tMap
  \end{itemize}
\end{slide}

\begin{slide}
  \begin{itemize}
    \item
      class Functor a where\\
      ~~fmap :: Functor f => (a -> b) -> f a -> f b
    \item
      instance Functor Maybe where\\
      ~~fmap :: (a -> b) -> Maybe a -> Maybe b\\
      ~~fmap \_ Nothing = Nothing\\
      ~~fmap f Just x = Just (f x)
    \item possiblyDouble m = fmap double m\\
      ~~~~where double x = x + x
  \end{itemize}
\end{slide}

\begin{slide}
  \begin{itemize}
    \item Recall Currying...
    \item plusX m = fmap plux m\\
      ~~~~where plux x = (+ x)
    \item uh oh...
    \item ghci> :t plusX [1,2,3,4,5]\\
      plusX [1,2,3,4,5] :: Num a => [a -> a]
  \end{itemize}
\end{slide}

\subsection{Applicative Functors}

\begin{slide}
  \begin{itemize}
    \item ghci> :t plusX [1,2,3,4,5]\\
      plusX [1,2,3,4,5] :: Num a => [a -> a]
    \item fmap again?
    \item
      class (Functor f) => Applicative f where\\
      ~~pure :: a -> f a\\
      ~~(<*>) :: f (a -> b) -> f a -> f b
    \item
      plusX [1,2,3,4] <*> [10]
  \end{itemize}
\end{slide}

\begin{slide}
  \begin{itemize}
    \item
      instance Applicative Maybe where\\
      ~~pure = Just\\
      ~~Nothing <*> \_ = Nothing\\
      ~~(Just f) <*> something = fmap f something
    \item pure (+3) <*> Just 9
    \item Just (++"hahah") <*> Nothing
    \item Nothing <*> Just "woot"
    \item pure (+) <*> Just 3 <*> Just 5
  \end{itemize}
\end{slide}

\begin{slide}
  \begin{itemize}
    \item
      instance Applicative [] where\\
      ~~pure = []\\
      ~~[] <*> \_ = []\\
      ~~(f:fs) <*> xs = (fmap f xs) ++ (fs <*> xs)
    \item pure (+) <*> pure 1 <*> pure 3
    \item Just (++"hahah") <*> Nothing
    \item Nothing <*> Just "woot"
    \item pure (+) <*> Just 3 <*> Just 5
  \end{itemize}
\end{slide}

\subsection{Monoids}

\begin{slide}
  Monoids
  \begin{itemize}
    \item
      class Monoid a where\\
      ~~mempty :: Monoid a => a\\
      ~~mappend :: Monoid a => a -> a -> a\\
      ~~mconcat :: Monoid a => [a] -> a
    \item
      instance Monoid [a] where\\
      ~~mempty = []\\
      ~~mappend = (++)\\
      ~~mconcat = foldr mappend mempty
  \end{itemize}
\end{slide}

\begin{slide}
  \begin{itemize}
    \item
      instance (Monoid a) => Monoid (Maybe a) where\\
      ~~mempty = Nothing

      ~~mappend Nothing x = x\\
      ~~mappend x Nothing = x\\
      ~~mappend (Just x) (Just y)\\
      ~~~~~~= Just (x \`{}mappend\`{} y)
  \end{itemize}
\end{slide}

\subsection{Future Topics}

\begin{slide}
  \begin{itemize}
    \item ``newtype'', ``type'', (\$), and Record Syntax
    \item Numeric Type Heirarchy and ByteStrings
    \item Monads
    \item Common Monads (I/O, Reader, Writer, State)
    \item MonadTransformers and MonadPlus
    \item Arrows
    \item Parser Combinators
    \item Category Theory and Advanced Types
    \item Advanced Functional Data Structures
      \begin{itemize}
        \item Trees as Maps
        \item Zippers
        \item Finger Trees (will blow your mind)
      \end{itemize}
  \end{itemize}
\end{slide}

\begin{slide}
  \begin{itemize}
    \item Testing/Quickcheck
    \item Error Handling
    \item Mutable Objects and Arrays
    \item Parallel Programming and STM
    \item Functional Reactive Programming
    \item Foreign Function Interface
    \item Cabal and Hackage
    \item Examine Real Code
      \begin{itemize}
        \item XMonad Window Manager (<1000 lines)
        \item Parsec
        \item Yesod Web Framework
        \item Darcs, Version Control System
        \item The Glorious Glasgow Haskell Compiler
      \end{itemize}
    \item Agda Theorem Prover
  \end{itemize}
\end{slide}




% - - - - - - - - - - - - - - - - - - - - - - - - - - - - - - - - - - - - - - -
% Where to go to learn more
% - - - - - - - - - - - - - - - - - - - - - - - - - - - - - - - - - - - - - - -

\renewcommand{\thefootnote}{\arabic{footnote}} % numbered footnotes
\setcounter{footnote}{0}                       % reset footnote numbering
\titleslide{Where to go to Learn More}
\begin{slide}
  \begin{itemize}
    \normalsize

  \item Learn You a Haskell for Great Good!
    \footnote{
      HTML:\\ \url{http://learnyouahaskell.com/}
    }

  \item Real World Haskell
    \footnote{
      HTML:\\ \url{http://book.realworldhaskell.org/}
    }

  \item Haskell Wikibook
    \footnote{
      HTML:\\ \url{https://secure.wikimedia.org/wikibooks/en/wiki/Haskell}
    }

  \item Write Yourself a Scheme in 48 Hours
    \footnote{
      HTML:\\ \url{http://halogen.note.amherst.edu/~jdtang/scheme_in_48/tutorial/overview.html}
    }

  \end{itemize}
\end{slide}

\renewcommand{\thefootnote}{\arabic{footnote}} % numbered footnotes
\setcounter{footnote}{0}                       % reset footnote numbering
\begin{slide}
  \begin{itemize}
    \normalsize

  \item Hackage Package Repository
    \footnote{
      HTML:\\ \url{http://hackage.haskell.org/}
    }

  \item Hoogle API Search
    \footnote{
      HTML:\\ \url{http://haskell.org/hoogle/}
    }

  \item Hayoo API Search
    \footnote{
      HTML:\\ \url{http://holumbus.fh-wedel.de/hayoo/hayoo.html}
    }

  \item \#Haskell IRC Channel
    \footnote{
      HTML:\\ \url{http://www.haskell.org/haskellwiki/IRC_channel}
    }

  \end{itemize}
\end{slide}

\titleslide{Thanks}
\begin{slide}
  Latex template and tech support:
  \begin{itemize}
    \item Daniel ``paradigm'' Thau
  \end{itemize}
  For teaching me Haskell and providing some of my examples:
  \begin{itemize}
    \item BONUS, Learn You a Haskell
    \item Bryan O'Sullivan, Don Stewart, and John Goerzen, Real World Haskell
  \end{itemize}
\end{slide}


\end{document}
